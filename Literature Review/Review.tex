\documentclass[12pt, letterpaper]{article}
\usepackage[utf8]{inputenc}
\usepackage{natbib}

\title{Literature Review}

\author{Shreyas Kalvankar \and Hrushikesh Pandit \and Pranav Parwate \and Atharva Patil}

\begin{document}
	\maketitle
	
	\begin{abstract}
		Astronomical imaging has made stupendous progress through the years with more and more sophisticated techniques being introduced with every new telescope launch. The rise in computation power enabled astronomers to increase the number of observations and resulted in flooding of information that could be processed as quickly as it could be seen by the human eye. The data albeit abundant, is computationally expensive to process and remains dormant in the archives globally. One such example being the Hubble Legacy Archive where there are hundreds and thousands of snapshots taken by the Hubble Space telescope which lie unprocessed in the database but can be of astronomical significance. This document summarizes a brief study of generative adversarial networks and convolutional neural networks and different techniques belonging to the disciplines that can be applied to astronomical images to make them usable for astronomical inspection. 
	\end{abstract}
	
	\section{Introduction}
		\hspace*{0.25 in}Hubble Legacy Archive or HLA is a project which endeavours to complement the Hubble Space Telescope by augmenting the HST Data Archive and providing superior browsing and searching capabilities. A large amount of raw images remain unprocessed in the HLA, never seen by a human eye. These raw images are typically low resolution, black and white and unfit to work with in today's day and age. It takes hundreds of hours to process them.
	\section{Hubble Legacy Archive}
		\hspace*{0.25 in}In the past decade, astronomical research was extensively performed using large catalogs which were search-able. This was made possible due to advances in computer technology and databases. The biggest challenge that is faced today is to convert this  large unstructured database into a comprehensive catalog. Advances in relational databases technology has made it efficient to create and store and search large catalogs. 

Sloan Digital Sky Survey or SDSS, was one such catalog. It uniformly observed the regions of the sky in a certain filter band at regular intervals of time. The HST however, has targeted only particular sources.

The Hubble Space Telescope (often referred to as HST or Hubble) is a space telescope that was launched into low Earth orbit in 1990 and remains in operation. It was not the first space telescope, but it is one of the largest and most versatile, well known both as a vital research tool and as a public relations boon for astronomy.

The Hubble Space Telescope archive was an archive for the Hubble Space Telescope which was launched in low Earth orbit back in 1990. It is an important tool for research and is the one of the biggest and most versatile telescopes which is active today.
The Hubble Legacy Archive (HLA) endeavors to create calibrated scientific data from the Hubble Space Telescope archive and make them accessible via user-friendly and Virtual Observatory (VO) compatible interfaces. 

	\section{Image Colorization}
		\hspace*{0.25 in} Automated colorization of gray scale images has been researched extensively throughout the machine learning community and is more specifically studied by those who indulge in the discipline of computer vision. Apart from being visually fascinating, it has many other applications ranging from restoration to enhancement for better interpretability. \\
		\hspace*{0.25 in}Owing to recent advances, the Convolutional Neural Networks are a de facto standard for solving image classification problems and their popularity continues to rise with continual improvements. CNNs are peculiar in their ability to learn and differentiate colors, patterns and shapes within an image and their ability to associate them with different classes. \\
		\hspace*{0.25 in} \cite{dahl2016automatic} successfully implemented a system to automatically colorize black \& white images using several ImageNet-trained layers from VGG-16 \cite{simonyan2015deep} and integrating them with auto-encoders that contained residual connections. These residual connections merged the outputs produced by the encoding VGG16 layers and the decoding portion of the network in the later stages. \cite{he2015deep} showed that deeper neural networks can be trained by reformulating layers to learn residual function with reference to layer inputs. Using this \textit{Residual Connections}, \cite{he2015deep} created the \textit{ResNets} that went as deep as 152 layers and won the 2015 ImageNet Challenge. 
		
\renewcommand\bibname{References}
\bibliographystyle{apalike}
\bibliography{References}

\end{document}