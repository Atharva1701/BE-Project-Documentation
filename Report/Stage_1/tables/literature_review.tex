\begin{table}[!htbp]
\begin{center}
\def\arraystretch{1.5}
  \begin{tabularx}{\textwidth}{| X | X | X | } \hline
Publication and Year & Technology & Summary \\ \hline
Cheng, Z., Yang, Q., and Sheng, B. (2016) &	Deep Colorization &	 The paper presented a fully-automatic colorization method using deep neural networks\\ \hline

Dahl, R. (2016) &	Automatic Colorization & automatically produce multiple colorized versions of a grayscale image\\ \hline

Goodfellow, I. J., Pouget-Abadie, J., Mirza, M., Xu, B., Warde-Farley, D., Ozair, S., Courville, A., and Bengio, Y. (2014) & Generative Adversarial Networks & Proposed a novel approach of implementing Generative Adversarial Networks using two Neural Networks, viz Generator and Discriminator Networks.\\ \hline

He, K., Zhang, X., Ren, S., and Sun, J. (2015) & Deep residual learning for image recognition & Presented 152 layer using residual learning framework for image recognition and an adaptive edge detection based colorization algorithm and its applications.\\ \hline

Isola, P., Zhu, J.-Y., Zhou, T., and Efros, A. A. (2018) & Image-to-image translation with conditional adversarial networks &  Pix2Pix is a Conditional-GAN with images as the conditions for colorization.\\ \hline

Ledig, C., Theis, L., Huszar, F., Caballero, J., Cunningham, A., Acosta, A., Aitken, A., Tejani, A., Totz, J., Wang, Z., and Shi, W. (2017) &  Super Resolution using GAN & Photorealistic single image super-resolution using a generative adversarial network.\\ \hline

Levin, A., Lischinski, D., and Weiss, Y. (2004) &  Colorization using optimization & Used quadratic cost function and were able to generate high quality colorizations. \\ \hline

Long, J., Shelhamer, E., and Darrell, T. (2015) &  Fully convolutional networks for semantic segmentation &  Showed that convolutional networks by themselves, trained end-to-end, pixels-to-pixels, improve on the previous best result in semantic segmentation.\\ \hline

\end{tabularx}
\caption{Literature Review}
 \label{tab:hreq}
\end{center}

\end{table}
\begin{table}[!htbp]
\begin{center}
\def\arraystretch{1.5}
  \begin{tabularx}{\textwidth}{| X | X | X | } \hline
Publication and Year & Technology & Summary \\ \hline

Mirza, M. and Osindero, S. (2014) &  Conditional generative adversarial nets & Introduced the conditional version of generative adversarial nets, which can be constructed by simply feeding the data, y, to condition on to both the generator and discriminator\\ \hline

Qu, Y., Wong, T.-T., and Heng, P.-A. (2006) &  Manga colorization & Proposed a novel colorization technique that propagates color over regions exhibiting pattern-continuity as well as intensity-continuity\\ \hline

Radford, A., Metz, L., and Chintala, S. (2016) &  Unsupervised representation learning with deep convolutional generative adversarial networks & Introduced a class of CNNs called deep convolutional generative adversarial networks (DCGANs), that have certain architectural constraints, and demonstrate that they are a strong candidate for unsupervised learning\\ \hline

Simonyan, K. and Zisserman, A. (2015) & Very deep convolutional networks for large-scale image recognition & Investigated the effect of the convolutional network depth on its accuracy in the large-scale image recognition setting\\ \hline

Tola, E., Lepetit, V., and Fua, P. (2008) & A fast local descriptor for dense matching & Introduced a novel local image descriptor designed for dense wide-baseline matching purposes \\ \hline

Tom and Katsaggelos (1996) & Reconstruction of a high-resolution image by simultaneous registration, restoration, and interpolation of low-resolution images & Solution is provided to the problem of obtaining a high resolution image from several low resolution images that have been subsampled and displaced by different amounts of sub-pixel shifts \\ \hline

\end{tabularx}
\caption{Literature Review(contd.)}
 \label{tab:hreq}
\end{center}
\end{table}
\begin{table}[!htbp]
\begin{center}
\def\arraystretch{1.5}
  \begin{tabularx}{\textwidth}{| X | X | X | } \hline
Publication and Year & Technology & Summary \\ \hline

TSAI, R. (1984) & Multiframe image restoration and registration & Applied and evaluated the ScSR method for improvement of image quality of magnified MR images (T1-weighted, T2-weighted, FLAIR, and DWI images) in16-bit DICOM format \\ \hline

Welsh, T., Ashikhmin, M., and Mueller, K. (2002) & Transferring color to greyscale images &  Introduced a general technique for colorizing greyscale images by transferring color between a source, color image and a destination, greyscale image \\ \hline

Yatziv, L. and Sapiro, G. (2006) & Fast image and video colorization using chrominance blending & High Quality colorization results are obtsined at a fraction of the complexity and computational cost using concepts of luminance-weighted chrominance blending and fast intrinsic distance computations \\ \hline

Zhu, J.-Y.,Krähenbühl, P., Shechtman, E., and Efros, A. A. (2018) & Generative visual manipulation on the natural image manifold & Defined a class of image editing operations ,after learning natural image manifold from data using generative adverserial neural networks, and constrain their output to lie on that learned manifold at all times \\ \hline
\end{tabularx}
\caption{Literature Review(contd.)}
 \label{tab:hreq}
\end{center}

\end{table}
